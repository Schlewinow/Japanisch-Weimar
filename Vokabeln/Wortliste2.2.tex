\noindent
\centering
Japanisch 2.2 Wortliste

Lektion 31

\begin{multicols}{2}
\begin{flushleft}
\begin{labeling}{ExampleWordLength}
	\item [\ruby{動物}{どうぶつ}] Tier
	\item [\ruby{象}{ぞう}] Elefant
	\item [\ruby{鼠}{ねずみ}] Maus, Ratte
	\item [\ruby{太陽}{たいよう}] Sonne, Hoffnung
	\item [\ruby{明}{あか}るい] hell, heiter
	\item [\ruby{硬}{かた}い] hart, fest, zuverlässig
	\item [\ruby{果物}{くだもの}] Obst, Frucht
	\item [\ruby{消}{け}しゴム] Radiergummi
	
	\item [しかし] aber, jedoch
	\item [もっと] mehr
	\item [\ruby{道路}{どうる}] Straße, Weg
	\item [\ruby{込}{こ}む] überfüllt werden
	\item [\ruby{特急電車}{とっきゅうでんしゃ}] Expresszug
	\item [\ruby{料金}{りょうきん}] Gebühr, Tarif
	\item [みんな] alle (anwesende Personen)
	\item [\ruby{働}{はたら}く] arbeiten, angestellt sein
	\item [\ruby{世界}{せかい}] Welt, Erde
	\item [\ruby{人気}{にんき}] Beliebtheit, Popularität
\end{labeling}
\end{flushleft}
\end{multicols}

Lektion 32

\begin{multicols}{2}
\begin{flushleft}
\begin{labeling}{ExampleWordLength}
	\item [ハンサムな] gut aussehend, hübsch
	\item [\ruby{真面目}{まじめ}な] ernst, ernsthaft
	\item [かっこいい] schick, toll, klasse
	\item [\ruby{大変}{たいへん}な] furchtbar, entsetzlich
	\item [\ruby{若}{わか}い] jung, jugendlich
	\item [\ruby{体}{からだ}にいい] gesund sein, gut für Gesundheit sein
	\item [\ruby{酸}{す}っぱい] sauer, herb
	\item [\ruby{苦}{にが}い] bitter, herb, sauer
	\item [\ruby{柔}{やわ}らかい] weich, sanft, mild
	\item [お\ruby{化}{ば}け] Gespenst, Schreckbild
	\item [\ruby{怖}{こわ}い] fürchterlich, schrecklich
	\item [サイコロ] Würfel
	\item [お\ruby{風呂}{ふろ}] Bad, Badewasser
	\item [\ruby{気持}{きも}ちいい] sich gut anfühlen, gut tun
\end{labeling}
\end{flushleft}
\end{multicols}

Lektion 33

\begin{multicols}{2}
\begin{flushleft}
\begin{labeling}{ExampleWordLength}
	\item [なる] werden
	\item [\ruby{氷}{こおり}] Eis
	\item [\ruby{医者}{いしゃ}] Arzt
	\item [\ruby{旅行}{りょこう}] Reise
	\item [\ruby{出発}{しゅっぱつ}] Abfahrt
	\item [\ruby{不便}{ふべん}な] unpraktisch
	\item [\ruby{眠}{ねむ}る] einschlafen, schlafen
	\item [すやすや\ruby{眠}{ねむ}る] friedlich schlafen
	\item [\ruby{回}{まわ}る] kreisen, drehen, rotieren
	\item [くるくる\ruby{回}{まわ}る] mehrfach auf Stelle drehen
	\item [お\ruby{腹}{なか}] Bauch, Magen
	\item [\ruby{一杯}{いっぱい}な] voll, überfüllt
	
	\item [ダイエット] Diät
	\item [\ruby{太}{ふと}る] dick werden, zunehmen
	\item [\ruby{本当}{ほんとう}に] wirklich, in der Tat
	\item [びっくりする] schockiert sein, überrascht sein
	\item [ふつう] normalerweise
	\item [だいたい] meistens
	\item [\ruby{大抵}{たいてい}] meist, gewöhnlich
	\item [\ruby{自分}{じぶん}] selbst, ich, eigen
	\item [とくに] besonders, vor allem
	\item [\ruby{痩}{や}せる] dünn werden, abnehmen
	\item [\ruby{人魚}{にんぎょ}] Meerjungfrau, Fischfrau
	
	\item [\ruby{商店街}{しょうてんがい}] Einkaufsstraße
	\item [\ruby{病気}{びょうき}] Krankheit, Leiden
	\item [\ruby{将来}{しょうらい}] Zukunft, kommende Zeit
	\item [\ruby{歌手}{かしゅ}] Sänger
\end{labeling}
\end{flushleft}
\end{multicols}

Lektion 34

\begin{multicols}{2}
\begin{flushleft}
\begin{labeling}{ExampleWordLength}
	\item [\ruby{又}{また}] wieder, erneut
	\item [\ruby{勝}{か}つ] gewinnen, siegen
	
	\item [\ruby{外国}{がいこく}] Ausland
	\item [\ruby{引}{ひ}っ\ruby{越}{こ}す] umziehen
	\item [\ruby{引}{ひ}っ\ruby{越}{こ}して\ruby{来}{く}る] einziehen
	\item [\ruby{大家}{おおや}さん] Vermieter, Hausbesitzer
	\item [\ruby{留学生}{りゅがくせい}] Auslandsstudent
	\item [\ruby{廊下}{ろうか}] Flur, Korridor
	\item [\ruby{声}{こえ}] Stimme
	\item [\ruby{嬉}{うれ}しい] glücklich, erfreut
	
	\item [\ruby{社長}{しゃちょう}] Firmenchef, Boss
	\item [\ruby{叫}{さけ}ぶ] schreien, rufen
\end{labeling}
\end{flushleft}
\end{multicols}

Lektion 35

\begin{multicols}{2}
\begin{flushleft}
\begin{labeling}{ExampleWordLength}
	\item [\ruby{遊}{あそ}びに\ruby{来}{く}る] besuchen, vorbeischauen
	\item [\ruby{祈}{いの}り] Gebet
	\item [\ruby{祈}{いの}りをする] beten
	\item [\ruby{風邪}{かぜ}を\ruby{引}{ひ}く] sich erkälten
	\item [\ruby{急}{いそ}ぐ] sich beeilen, hasten
	\item [\ruby{遅}{おく}れる] verpassen, verspäten
	\item [\ruby{間違}{まちが}う] Fehler machen, falsch liegen
	\item [\ruby{病気}{びょうき}を\ruby{治}{なお}す] Krankheit heilen
	\item [\ruby{休}{やす}む] Pause machen, ausruhen, etwas nicht machen
	\item [お\ruby{金}{かね}を\ruby{下}{お}ろす] Geld abheben
	\item [\ruby{下}{お}ろす] herablassen, herunterholen
	\item [お\ruby{金}{かね}を\ruby{預}{あず}ける] Geld einzahlen
	\item [\ruby{預}{あず}ける] anvertrauen, zur Aufbewahrung geben
	\item [\ruby{閉}{と}じる] schließen, zumachen (z. Bsp. Augen, Buch)
	\item [\ruby{落}{お}ちる (\ruby{落}{お}ちます)] fallen, durchfallen
	\item [\ruby{化粧}{けしょう}] Schminke, Kosmetik
	\item [\ruby{化粧}{けしょう}をする] schminken
	
	\item [\ruby{始}{はじ}まる] beginnen, anfangen
	\item [\ruby{終}{お}わる] enden, beenden
	\item [\ruby{残業}{ざんぎょう}] Überstunden
	\item [\ruby{洗濯}{せんたく}] Wäsche (nominalisiertes waschen)
	\item [\ruby{洗濯}{せんたく}をする] Wäsche waschen
	\item [\ruby{出}{で}かける] ausgehen, losgehen
	
	\item [\ruby{貯}{た}める] (Geld) sparen
	\item [\ruby{忘}{わす}れる] vergessen
\end{labeling}
\end{flushleft}
\end{multicols}

Lektion 36

\begin{multicols}{2}
\begin{flushleft}
\begin{labeling}{ExampleWordLength}
	\item [\ruby{大切}{たいせつ}] wichtig, bedeutsam, kostbar
	\item [\ruby{体}{からだ}] Körper, Leib, Statur
	\item [\ruby{吃驚}{びっくり}する] sich erschrecken, überrascht sein
	\item [\ruby{未}{ま}だ] noch, immer noch
	\item [\ruby{最近}{さいきん}] in letzter Zeit, heutzutage
	\item [\ruby{以前}{いぜん}] früher, vor, bevor
	\item [\ruby{何度}{なんど}か] öfter, mehrfach
	\item [\ruby{小説}{しょうせつ}] Roman, (Kurz)Geschichte
	\item [\ruby{全部}{ぜんぶ}] alles, ganz
	\item [\ruby{覚}{おぼ}える] merken, einprägen
	\item [\ruby{無理}{むり}] Unmöglichkeit, Unvernunft
	\item [\ruby{集}{あつ}める] sammeln, versammeln, einsammeln
	\item [\ruby{上}{のぼ}る] besteigen, erklettern
	\item [\ruby{捕}{つか}まる] verhaftet werden, gefasst werden
	\item [\ruby{喧嘩}{けんか}] Streit, Prügelei
	\item [\ruby{喧嘩}{けんか}する] streiten, zanken
	
	\item [\ruby{経験}{けいけん}] Erfahrung
	\item [\ruby{経験}{けいけん}する] eine Erfahrung machen、etwas erfahren
	\item [\ruby{夕方}{ゆうがた}] früher Abend
	\item [\ruby{並}{なら}ぶ] in Reihe aufstellen
	\item [\ruby{空}{す}く] leer werden, ausdünnen
	\item [お\ruby{客}{きゃく}さん] Kunde, Gast
	\item [\ruby{間違}{まちが}える] Fehler machen, irren
	\item [\ruby{一生懸命}{いっしょうけんめい}] emsig, eifrig, mit ganzer Kraft
	\item [\ruby{始}{はじ}めは] am Anfang, zu Beginn
	\item [\ruby{疲}{つか}れる] müde werden, ermüden
	\item [\ruby{止}{や}める] (eine Tätigkeit) aufhören, abbrechen, beenden
	\item [\ruby{高校}{こうこう}] Highschool
	\item [\ruby{高校生}{こうこうせい}] Highschool-Schüler
	\item [\ruby{禁止}{きんし}] Verbot
\end{labeling}
\end{flushleft}
\end{multicols}

\clearpage